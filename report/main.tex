\documentclass[runningheads]{llncs}
\usepackage[utf8]{inputenc}
\usepackage{setspace}
\usepackage{amssymb}
\usepackage{subfiles}
\usepackage{amsmath}
\usepackage[
backend=biber,
style=alphabetic,
giveninits=true
]{biblatex}
\DeclareNameAlias{default}{family-given}
\addbibresource{sources.bib}

\newcommand{\GCP}{{Graph Coloring Problem}\ }

\title{Surveying Cooperative Approaches for the Graph Coloring Problem}
\author{Gabriel Perez
\and
Corey Roberts
\and
Javier Jesus Valenzuela
}
\date{\today}

\authorrunning{Perez and Valenzuela and Roberts}

\institute{ Department of Electrical and Computer  Engineering\\
  University of Texas at Austin,\\
  Austin, TX 78712\\
 \email{\{email1, email2\}@utexas.edu}}

\begin{document}

\def\IEEEQED{\mbox{\rule[0pt]{1.3ex}{1.3ex}}} % for a filled box
\newcommand{\ep}{\hspace*{\fill}~\IEEEQED}
\newenvironment{mproof}[1][Proof]{{\bf #1: }}{\ep\vspace{.1in}}

\maketitle
\doublespacing

\begin{abstract}
The \GCP is an NP-Complete problem with a rich history of approximation algorithms and heuristics. In this paper, we explore a modification to cooperative local search approaches for the graph coloring problem.
\end{abstract}

\section{Introduction}
The \GCP and its variants are long existing problems with a very low barrier to understanding. Despite that, the class of problems are incredibly difficult problem to solve exactly. Applications of the problems include identifying whether a circuit board has a short circuit, solving Sudoku for any size grid, and problems of scheduling exams for university lecturers \cite{10.5555/2851123}.

In this paper, we explore the idea of multiple algorithms cooperating to determine whether we get improved lower bounds for the \GCP approximations.

\section{Definitions}

\subsection{Graph \& Graph Properties}

A graph, $G$, is defined as $G = (V, E)$ a set of $n$ vertices $V$ and $m$ edges $E$. For this class of problems, we have a function $c: V \mapsto \mathbb{N} $. This function is called the \emph{coloring function}. The coloring is \emph{legal} if $\forall (u, v) \in E \implies c(u) \ne c(v)$. If the graph can be colored by using $k$ colors, then the graph is called $k-colorable$. The minimum $k$ that can be used to color the graph $G$ is called the \emph{chromatic number} of the graph $G$ denoted $\chi$. Another important term is neighboring coloring, which involves changing the color of a single vertex.

\section{Existing Graph Coloring Problem Approaches }

In this section of the paper, we will reference some existing techniques and heuristics for graph coloring as they are vital to understanding the hypothesis.

\subsection{Tabucol \& HCD}\label{AA}
Tabucol is an algorithm proposed by Hertz and de Werra \cite{hdw}. The implementation here is the modified version by Galinier and Hao \cite{tabucol} and discussed in the \emph{Guide to Graph Coloring} textbook \cite{10.5555/2851123}. The algorithm fundamentally revolves around moves, i.e., shifting a coloring to a neighboring coloring by only \emph{moving} the vertex causing the most conflicts to another color class. The algorithm attempts to prevent cyclic issues by making certain moves \emph{taboo} hence the name. The rate at which moves become accessible again is dependent upon algorithm parameters. Another algorithm that was used in this paper is called HCD. HCD itself is very similar to the tabu approaches, but uses a priority based system. Further references can be found in the referenced paper \cite{10.1007/3-540-48318-7_25}.


\subsection{Metropolis Algorithm}
The metropolis algorithm is a very simple algorithm in which we select vertices, color them, analyze their impact to a cost function $H$ (how many neighbors are similarly colored), and approve or reject them with a probability based on the move itself. This algorithm also comes with hyper-parameters that are adjustable.


\section{Cooperative Extensions to the Graph Coloring Problem}

This paper is comprised of several experiments:

\begin{itemize}

  \item Assessing whether one sees improved lower bounds on $\chi$ by running concurrent metropolis algorithms sharing information.
  \item Assessing the performance improvements of of PartialCol, HCD, and Tabucol sharing lower bounds and partial colorings in identifying lower $\chi$.
    \item Assessing cooperative Tabucol approaches with other algorithms - that is to say extending the results of Li. \cite{https://doi.org/10.5445/ir/1000083192}
\end{itemize}

The source code is available in this project. It was written using a standard Java console application with gradle for managing dependencies. The most important of those dependencies is JGraphT, an open source library for handling graphs. The graph format that was used is the popular DIMACS format.

\subsection{Cooperative Metropolis Algorithm}





% \subsection{HCD}\label{AA}

% Another algorithm that was used in this paper is called HCD. HCD itself is very similar to the tabu approaches, but uses a priority based system. It is broken into four phases. The first is an initialization phase where the colors are assigned based on vertex numbers. The second is pulling where HCD assigns the lowest color value compatible
% with those assigned to its neighbors to the vertex with the lowest priority. Third is pushing where each vertex is assigned the highest color feasible that still works with the arrangement. The final stage is popping, whose objective is to try and solve local optima issues that frequently arise with these heuristics.

\printbibliography

\end{document}
